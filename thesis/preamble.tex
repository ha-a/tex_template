% !TEX root = main.tex

\usepackage{comment}

%%%%%%%%%%%%%%
% TITLE PAGE %
%%%%%%%%%%%%%%

\makeatletter

%\def\@thesis{特別研究}
\def\thesis#1{\def\@thesis{#1}}
\def\id#1{\def\@id{#1}}
\def\department#1{\def\@department{#1}}
\def\entitle#1{\def\@entitle{#1}}

\def\@maketitle{
  \vbox{}\vfill%\vspace{40mm}

  \begin{center}
    {\LARGE \@thesis \par}
      \vspace{8mm}
    {\huge \@title \par}
      \vspace{2mm}
    {\Large \@entitle \par}
      \vspace{10mm}
    {\Large \@date\par}
      \vspace{20mm}
    {\large \@department \par}
      \vspace{5mm}
    {\large 学籍番号 \@id}\hspace{1cm}{\LARGE \@author}
  \end{center}

  \vbox{}\vfill
}

\makeatother

%%%%%%%%%%%%%%%%%
% MARGIN & PAGE %
%%%%%%%%%%%%%%%%%

\usepackage{geometry}
\geometry{
    left=2.5cm,
    right=2.5cm,
    top=3cm,
    bottom=3cm
}

\usepackage{fancyhdr}
%\pagestyle{fancy}

\usepackage{extramarks}
\pagestyle{fancy}
\fancyhf{}
\fancyhead[RO]{\lastxmark}
\fancyhead[LE]{\leftmark}
%\fancyfoot[c]{\thepage}
%\renewcommand*{\sectionmark}[1]{\extramarksleft{\thesection.\ #1}}
%\renewcommand*{\subsectionmark}[1]{\extramarksright{\thesubsection.\ #1}}

\renewcommand{\sectionmark}[1]{\markboth{\thesection.\ #1}{\thesection.\ #1}\extramarks{}{\thesection.\ #1}}
\renewcommand{\subsectionmark}[1]{\extramarks{}{\thesubsection.\ #1}}

\fancyhf{}
\fancyhead[LE]{\leftmark}
\fancyhead[RO]{\lastxmark}

%\fancyhf{}
% 片面印刷
%\rhead{\nouppercase{\leftmark}}
%\lhead{}
%\cfoot{\thepage}

% 両面印刷
%\fancyhead[LE]{\leftmark}
%\fancyhead[RO]{\rightmark}
\fancyfoot[LE,RO]{\thepage}


\fancypagestyle{toc}{
  \fancyhead{}
  \fancyfoot[LE,RO]{\thepage}
  \renewcommand{\headrulewidth}{0pt}
}


%%%%%%%%%%%%%%%%%
% CONTENTS NAME %
%%%%%%%%%%%%%%%%%

\usepackage{titlesec}
\renewcommand{\contentsname}{Table of Contents}
\renewcommand{\listfigurename}{List of Figures}
\renewcommand{\listtablename}{List of Tables}
\renewcommand{\figurename}{図}
\renewcommand{\tablename}{表}
\renewcommand{\refname}{References}
\renewcommand{\appendixname}{付録}

\renewcommand{\thesection}{\arabic{section}}
\renewcommand{\thesubsection}{\thesection.\arabic{subsection}}
\renewcommand{\thesubsubsection}{(\arabic{subsubsection})}


%%%%%%%%%%%%%%%%%
% SECTION STYLE %
%%%%%%%%%%%%%%%%%

\titleformat{\section} % command
  [hang]               % shape
  {\bfseries\LARGE}    % format
  {\thesection.}       % label
  {0.5em}              % sep
  {}                   % before-code
 
\titleformat{\subsection} % command
  [hang]                  % shape
  {\bfseries\Large}       % format
  {\thesubsection.}       % label
  {0.5em}                 % sep
  {}                      % before-code

\titleformat{\subsubsection} % command
  [hang]                     % shape
  {\bfseries\large}          % format
  {\thesubsubsection}        % label
  {0.5em}                    % sep
  {}                         % before-code


%%%%%%%%%%%%%%%%%%%%%
% TABLE OF CONTENTS %
%%%%%%%%%%%%%%%%%%%%%

\setcounter{tocdepth}{3}

\usepackage{tocloft}
% \renewcommand{\cftdot}{$\ast$}
\renewcommand{\cftsecfont}{\normalsize}      % 見出しフォント
\renewcommand{\cftsecpagefont}{\normalsize}  % ページ番号フォント
\renewcommand{\cftsecpresnum}{}              % 見出し番号前
\renewcommand{\cftsecaftersnum}{.}           % 見出し番号後
\cftpagenumbersoff{subsection}               % ページ番号を表示しない
\renewcommand{\cftsubsecaftersnum}{.}
\cftpagenumbersoff{subsubsection}
\renewcommand{\cftsubsubsecpresnum}{}
\renewcommand{\cftsubsubsecaftersnum}{}

\renewcommand{\cftsecleader}{\cftdotfill{\cftsecdotsep}}  % 見出し....ページ番号
\renewcommand{\cftdotsep}{1}                              % 点線の点間の調整
\renewcommand{\cftsecdotsep}{\cftdotsep}
\setlength{\cftbeforesecskip}{2.5pt}                      % 見出し上の余白調整

\cftsetindents{section}{0em}{1.25em}
\cftsetindents{subsection}{1.5em}{2em}
\cftsetindents{subsubsection}{3em}{1.5em}

% LIST OF FIGURES
\cftpagenumbersoff{figure}
\renewcommand{\cftfigpresnum}{\figurename}  % 図番号の前
\renewcommand{\cftfigaftersnum}{.}          % 図番号の後
\renewcommand{\cftfignumwidth}{4em}         % 図見出しインデント
% 段組み
\RequirePackage{multicol}
\setlength{\columnsep}{-10pt}
\renewcommand\cftlofprehook{\begin{multicols}{2}\setlength{\columnseprule}{.4pt}}
\renewcommand\cftlofposthook{\end{multicols}}

% LIST OF TABLES
\cftpagenumbersoff{table}
\renewcommand{\cfttabpresnum}{\tablename}   % 表番号の前
\renewcommand{\cfttabaftersnum}{.}       % 表番号の後
\renewcommand{\cfttabnumwidth}{4em}   % 表見出しインデント

\RequirePackage{multicol}
\setlength{\columnsep}{-10pt}
\renewcommand\cftlotprehook{\begin{multicols}{2}\setlength{\columnseprule}{.4pt}}
\renewcommand\cftlotposthook{\end{multicols}}

%%%%%%%%%%%
% FIGURES %
%%%%%%%%%%% 

\usepackage{graphicx}
\graphicspath{
  {./figures}
  {./figures/section1/},%
  {./figures/section2/},%
  {./figures/Appendix/},%
}

% CAPTION
\makeatletter
  \newcommand{\figcaption}[1]{\def\@captype{figure}\caption{#1}}
  %\newcommand{\tblcaption}[1]{\def\@captype{table}\caption{#1}}
  \usepackage{xifthen}
  \newcommand{\tblcaption}[2][]{%
    \def\@captype{table}%
    \ifthenelse{\equal{#1}{}}{\caption{#2}}{\caption[#1]{#2}}
  }
\makeatother

\usepackage[list=true,labelformat=parens,subrefformat=simple]{subcaption}
\renewcommand\thesubfigure{(\alph{subfigure})}
\DeclareCaptionLabelFormat{opening}{#2}
\captionsetup[figure]{
  format=hang,
  labelsep=period,
  font=small,
  %justification=RaggedRight,
  singlelinecheck=1,
}
\captionsetup[table]{
  format=hang,
  labelsep=period,
  %justification=RaggedRight,
  singlelinecheck=1,
}
\captionsetup[subfigure]{
  labelformat=opening,
  format=hang, 
  justification=raggedright,
  singlelinecheck=0,
}

% 
\newcommand{\figref}[1]{\figurename\ref{#1}}
\newcommand{\tblref}[1]{\tablename\ref{#1}}

\usepackage[pdfusetitle]{hyperref}
\hypersetup{
  unicode,
  luatex,
  pdfencoding=auto,
  bookmarksnumbered=true,
  colorlinks=false,
  linkbordercolor={0 0 1},
  hidelinks,
}

%%%%%%%%%%%%%
% NUMBERING %
%%%%%%%%%%%%%

\makeatletter
  \renewcommand{\theequation}{\thesection.\arabic{equation}} 
  \@addtoreset{equation}{section}

  \renewcommand{\thefigure}{\thesection.\arabic{figure}}
  \@addtoreset{figure}{section}

  \renewcommand{\thetable}{\thesection.\arabic{table}}
  \@addtoreset{table}{section}
\makeatother


%%%%%%%%%%%%%%%
% MATHEMATICS %
%%%%%%%%%%%%%%%

\usepackage{amsmath}
\usepackage{amsfonts}
\usepackage{amssymb}
\usepackage{amsthm}
\usepackage{bm}
\usepackage{array}
\usepackage{mathtools} % \coloneqq

% 数式モードの行間
\setlength{\jot}{10pt}

\newenvironment{subeq}{%
  \subequations\renewcommand{\theequation}{\theparentequation\,\alph{equation}}
}{%
  \endsubequations
}

\newcommand{\diff}{\mathrm{d}}


%%%%%%%%%%%%%
% FUNCTIONS %
%%%%%%%%%%%%%

\newcommand{\alert}[1]{\textcolor{red}{\textbf{(#1)}}}
\newcommand{\pgref}[1]{\pageref{#1}ページ}
% \usepackage{newclude}
\usepackage{subfiles}

\usepackage{ascmac} % screen, itembox, shadebox
\usepackage{color}

\usepackage{enumitem}
\setlist{listparindent=\parindent}


%%%%%%%%%%%%%%%%%%
%%%%%%%%%%%%%%%%%%
%%%%%%%%%%%%%%%%%%

\newcommand{\Kn}[0]{\mathrm{Kn}}
\newcommand{\vw}[0]{v_\mathrm{w}}
\newcommand{\QV}[0]{Q_\mathrm{V}}

\renewcommand{\labelenumi}{\roman{enumi}.}
\renewcommand{\labelitemi}{$\circ$}

\newcommand{\circtext}[1]{\raise0.2ex\hbox{\textcircled{\scriptsize{#1}}}}


\newcommand{\capt}[3][\hsize-5em]{%
  \def\@captype{figure}%
  \caption[#2]{
    \begin{minipage}[t]{#1}
      \textbf{#2}\\
      #3
    \end{minipage}
  }
}



%%%%%%%%%%%%%%%%
% BIBLIOGRAPHY %
%%%%%%%%%%%%%%%%
% 何故かバグるのでここに置く

\usepackage[numbers,sort&compress]{natbib}
\renewcommand{\refname}{参考文献}
\bibliographystyle{unsrtnat}